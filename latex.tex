\documentclass[a4paper,12pt]{article}
\usepackage[slovene]{babel}
\usepackage[utf8]{inputenc}
\usepackage[T1]{fontenc}
\usepackage[a4paper, total={17cm, 23cm}]{geometry}
\usepackage{lmodern}
\usepackage{amsmath,amsfonts}
\usepackage{amsthm}
\usepackage{setspace}

\def\N{\mathbb{N}}
\def\Z{\mathbb{Z}}
\def\Q{\mathbb{Q}}
\def\R{\mathbb{R}}

\theoremstyle{definition}
\newtheorem{definicija}{Definicija}
\theoremstyle{plain}
\newtheorem{izrek}{Izrek}
\newtheorem{lema}{Lema}
%\newtheorem{zgled}{Zgled}

\newenvironment{dokaz}{\begin{proof}[\bfseries\upshape\proofname]}{\end{proof}}
\newenvironment{zgled}{\begin{proof}[\bfseries\upshape Zgled]}{\end{proof}}

\newcommand{\geslo}[2]{\noindent \textbf{#1} \quad #2 \hfill \break}

\setstretch{1.2}

\title{Cesaro zveznost in odvedljivost \\ Seminar}
\author{Matevž Miščič}
\date{2. april 2020}



\begin{document}

\maketitle{}

\section*{Uvod}
V prvem letniku smo spoznali, kako lahko zveznost funkcije v neki točki karakteriziramo z zaporedji. Funkcija $f: \mathbb{R} \rightarrow \mathbb{R}$ je zvezna v točki $a \in \mathbb{R}$ natanko tedaj, ko za vsako zaporedje $(a_n)$, ki konvergira proti $a$, zaporedje $(f(a_n))$ konvergira proti $f(a)$. Podobno lahko z zaporedji karakteriziramo funkcijsko limito: število $L \in \mathbb{R}$ je limita funkcije $f: \mathbb{R} \rightarrow \mathbb{R}$ v točki $a \in \mathbb{R}$ natanko tedaj, ko za vsako zaporedje $(a_n)$ s členi različnimi od $a$, ki konvergira proti $a$, zaporedje $(f(a_n))$ konvergira proti $L$. Po definiciji funkcija $f: \mathbb{R} \rightarrow \mathbb{R}$ odvedljiva v $a$, če obstaja limita $$\lim_{x \rightarrow a} \frac{f(x)-f(a)}{x-a},$$ v tem primeru tej limiti pravimo odvod funkcije $f$ v točki $a$. Ker funkcijsko limito znamo opisati z zaporedji, lahko rečemo, da je $f$ odvedljiva v $a$, natanko tedaj, ko obstaja število $L \in \mathbb{R}$, da za vsako zaporedje $(a_n)$ s členi različnimi od $a$, ki konvergira proti $a$, zaporedje $(\frac{f(x)-f(a_n)}{x-a_n})$ konvergira proti $L$. Sedaj smo uspeli pojma zveznosti in odvedljivosti funkcije opisati samo s pojmom konvergence zaporedja. Če bi znali tudi kako drugače definirati, kdaj dano zaporedje konvergira, bi dobili drugačni definiciji zveznosti in odvedljivosti. Točno s tem se bomo ukvarjali v tej predstavitvi. Najprej se bomo naučili, kaj je to Cesaro konvergenca zaporedja, s pomočjo tega bomo dobili novi definiciji zveznosti in odvedljivosti, nato pa bomo ugotovili katere funkcije so zvezne oziroma odvedljive po tej novi definiciji.



\section{Cesaro konvergenca}
Naj bo $(a_n)$ realno zaporedje. Temu zaporedju lahko priredimo novo zaporedje, katerega n-ti člen je enak $\overline{a}_n = \frac{a_1+a_2+\ldots+a_n}{n}$. Temu novemu zaporedju bomo rekli zaporedje aritmetičnih sredin zaporedja $(a_n)$ in ga označili z $(\overline{a}_n)$.
\begin{definicija}
    Realno zaporedje $(a_n)$ Cesaro konvergira, če konvergira njeno zaporedje aritmetičnih sredin $(\overline{a}_n)$.
\end{definicija}
Oznaka $a_n \rightarrow a$ naj pomeni, da zaporedje $(a_n)$ konvergira proti $a$, oznaka $a_n \leadsto a$ pa, da zaporedje Cesaro konvergira proti $a$. 

V prvem letniku smo se pri analizi naučili, da iz $a_n \rightarrow a$ sledi $a_n \leadsto a$. Obratno seveda ne velja, saj zaporedje $((-1)^n)$ Cesaro konvergira proti $0$, ne konvergira pa v običajnem smislu. Ker je pojem Cesaro konvergence zelo pomemben za nadaljevanje, si oglejmo še nekaj primerov.

\subsection*{Primeri}
\begin{enumerate}
    \item Poiščimo primer omejenega zaporedja, ki ni Cesaro konvergentno. Prvi člen naj bo enak $2$. Naslednjih nekaj členov bo enakih $-2$. Takih členov mora biti dovolj, da bo aritmetična sredina padla pod $-1$. Nato spet dodajmo dovolj členov enakih $2$, da bo aritmetična sredina narasla nad $1$. S ponavljanjem take konstrukcije dobimo omejeno zaporedje, ki ni Cesaro konvergentno, saj zaporedje aritmetičnih sredin nekako oscilira med $-1$ in $1$.
    \item Naj bo $m \in \mathbb{N}$ in $a_1, a_2, \ldots, a_m \in \mathbb{R}$. Zanima nas, kdaj zaporedje $a_1, a_2, \ldots, a_m, a_1, a_2, \ldots$ Cesaro konvergira proti $0$. Naj bo $A := a_1 + a_2 + \ldots + a_m$. Ker za vse $k \in \mathbb{N}$ velja $\overline{a}_{km} = A$, je enakost $A = 0$ potreben pogoj za $a_n \leadsto 0$. Naj bo torej $A = 0$. Zaporedje delnih vsot zaporedja $(a_n)$ je potem periodično, zato je omejeno. Sledi, da zaporedje aritmetičnih sredin konvergira proti $0$. Torej $(a_n)$ konvergira proti $0$ natanko tedaj, ko velja $A = 0$.
\end{enumerate}



\section{Cesaro zveznost}
Sedaj se lahko končno lotimo Cesaro zveznosti.

\begin{definicija}
    Funkcija $f: \mathbb{R} \rightarrow \mathbb{R}$ je Cesaro zvezna v točki $a \in \mathbb{R}$, če za vsako zaporedje $(a_n)$, ki Cesaro konvergira proti $a$, zaporedje $(f(a_n))$ Cesaro konvergira proti $f(a)$. Pravimo, da je $f$ zvezna, če je zvezna v vsaki točki $a \in \mathbb{R}$.
\end{definicija}

Da si bomo lažje predstavljali, katere funkcije so Cesaro zvezne, si najprej oglejmo kakšen primer.

\subsection*{Primeri}
\begin{enumerate}
    \item Pokazati želimo, da je vsaka funkcija oblike $f(x) = Ax + B$, kjer sta $A, B \in \mathbb{R}$, Cesaro zvezna. Naj bo $(a_n)$ poljubno konvergentno zaporedje in $a$ njegova limita. Zaporedje $(A a_n + B)$ Cesaro konvergira k $A a + B = f(a)$, torej je funkcija $f$ res Cesaro zvezna.
    \item Oglejmo si še primer funkcije, ki ni Cesaro zvezna. Naj bo $f(x) = x^2$ funkcija. Zaporedje $((-1)^n)$ Cesaro konvergira k $0$, zaporedje $(f((-1)^n))$ pa je konstantno enako $1$, zato Cesaro konvergira k $1$ in ne k $f(0) = 0$. Torej $f$ ni Cesaro zvezna v točki $0$.
\end{enumerate}

Kot smo ugotovili, je vsaka funkcija oblike $f(x) = Ax + B$ Cesaro zvezna. Izkaže se, da so to tudi vse Cesaro zvezne funkcije.

\begin{izrek}
    Naj bo $f: \mathbb{R} \rightarrow \mathbb{R}$ funkcija. Naslednje trditve so ekvivalentne.
    \begin{enumerate}
        \item Funkcija $f$ je Cesaro zvezna v točki $0$.
        \item Funkcija $f$ je Cesaro zvezna.
        \item Funkcija $f$ je oblike $f(x) = Ax + B$ za neki realni števili $A, B \in \mathbb{R}$.
    \end{enumerate}
\end{izrek}
\begin{dokaz}
    $(1) \Rightarrow (3): $ Naj bo funkcija $g: \mathbb{R} \rightarrow \mathbb{R}$ definirana s predpisom $g(x) = f(x) - f(0)$. Potem je $g$ Cesaro zvezna v točki $0$ in velja $g(0) = 0$. 
    Naj bo $a \in \mathbb{R}$ poljubno realno število. Ker zaporedje $a, -a, a, -a, a, \ldots$ Cesaro konvergira proti $0$ in je $g$ Cesaro zvezna v $0$, zaporedje $g(a), g(-a), g(a), g(-a), \ldots$ Cesaro konvergira proti $g(0) = 0$. Potem mora veljati $g(a) + g(-a) = 0$ oziroma $g(-a) = -g(a)$. 
    Naj bosta zdaj $b, c \in \mathbb{R}$ poljubni realni števili. Spet zaporedje $b, c, -(b+c), b, c, -(b+c), \ldots$ Cesaro konvergira k $0$, zato zaporedje $g(b), g(c), g(-(b+c)), g(b), \ldots$ Cesaro konvergira proti $g(0) = 0$. Sledi $g(b) + g(c) + g(-(b+c)) = 0$ oziroma $-(g(b) + g(c)) = g(-(b+c))$. Upoštevamo še, da je velja $g(-a) = -g(a)$ za vsak $a \in \mathbb{R}$ in dobimo $g(b) + g(c) = g(b+c)$. Torej je $g$ aditivna.

    Naslednji cilj je pokazati, da velja $g(\lambda x) = \lambda g(x)$ za vse $\lambda \in \mathbb{Q}$ in $x \in \mathbb{R}$. Za primer ko je $\lambda \in \mathbb{N}$ to sledi neposredno iz aditivnosti. Ker velja tudi $g(0) = 0$ in $g(-a) = -g(a)$, to velja celo za vse $\lambda \in \mathbb{Z}$. Naj bo zdaj $\frac{m}{n} \in \mathbb{Q}$ poljubno racionalno število. Velja $mg(x) = g(mx) = g(n\frac{m}{n}x) = ng(\frac{m}{n}x)$ oziroma $\frac{m}{n}g(x) = g(\frac{m}{n}x)$, kar smo želeli dokazati.

    Pokažimo zdaj, da je $g$ zvezna. Naj bo $(x_n)$ poljubno zaporedje, da je $x_n \rightarrow 0$. Poiščimo zaporedje $(y_n)$, katerega zaporedje aritmetičnih sredin je enako $(x_n)$. Očitno mora biti $y_1 = x_1$. Denimo, da smo že definirali $y_1, \ldots, y_n$ in da velja $x_k = \overline{y}_k$ za vse $k \leq n$. Da bo veljalo tudi $x_{n+1} = \overline{y}_{n+1}$ oziroma $x_{n+1} = \frac{y_1 + \ldots + y_{n+1}}{n+1}$, moramo vzeti $y_{n+1} = (n+1)x_{n+1} - (y_1 + \ldots + y_n)$. Tako definirano zaporedje $(y_n)$ res zadošča $x_n = \overline{y}_n$ za vse $n \in \mathbb{N}$. Ker je $x_n \rightarrow 0$, je $y_n \leadsto 0$ po definiciji, zato iz Cesaro zveznosti funkcije $g$ v $0$ sledi $g(y_n) \leadsto g(0) = 0$. Iz tega, kar smo pokazali v prejšnjih odstavkih, sledi 
    $$g(x_n) = g(\overline{y}_n) = g(\frac{y_1 + \ldots + y_n}{n}) = \frac{g(y_1) + \ldots + g(y_n)}{n} \rightarrow 0.$$
    Torej je $g$ zvezna v $0$. Ker je $g(x_0 + x) = g(x_0) + g(x)$, je zvezna tudi v vsaki drugi točki $x_0 \in \mathbb{R}$.

    Naj bo $A = g(1)$. Zvezni funkciji $g$ in $x \mapsto Ax$ se ujemata na $\mathbb{Q}$, ki je gosta podmnožica v $\mathbb{R}$, torej sta enaki. Če vzamemo $B = f(0)$, velja $f(x) = Ax + B$ za vse $x \in \mathbb{R}$. S tem je implikacija dokazana.

    $(3) \Rightarrow (2): $ To smo pokazali v zgornjem zgledu.

    $(2) \Rightarrow (1): $ Če je $f$ Cesaro zvezna, je po definiciji Cesaro zvezna tudi v točki $0$.
\end{dokaz}


\section{Cesaro odvedljivost}
\begin{definicija}
    Funkcija $f: \mathbb{R} \rightarrow \mathbb{R}$ je Cesaro odvedljiva v točki $a \in \mathbb{R}$, če obstaja število $f'(a) \in \mathbb{R}$, da za vsako zaporedje $(a_n)$ s členi različnimi od $a$, ki Cesaro konvergira proti $a$, zaporedje diferenčnih kvocientov $(\frac{f(a_n)-f(a)}{a_n-a})$ Cesaro konvergira proti $f'(a)$. Številu $f'(a)$ v takem primeru rečemo Cesaro odvod funkcije $f$ v točki $a$. Pravimo, da je $f$ odvedljiva, če je odvedljiva v vsaki točki $a \in \mathbb{R}$.
\end{definicija}

Oglejmo si nekaj primerov funkcij in poskusimo ugotoviti, če so Cesaro odvedljive.

\subsection*{Primeri}
\begin{enumerate}
    \item Naj bo $f: \mathbb{R} \rightarrow \mathbb{R}$ funkcija oblike $f(x) = Ax^2 + Bx + C$ za neke $A, B, C \in \mathbb{R}$. Naj bo $(a_n)$ poljubno Cesaro konvergentno zaporedje s Cesaro limito $a$. Velja 
    \begin{align*}
        \frac{f(a_n)-f(a)}{a_n-a} &= \frac{(Aa_n^2 + Ba_n + C)-(Aa^2 + Ba + C)}{a_n-a}\\
        &= \frac{A(a_n-a)(a_n+a) + B(a_n-a)}{a_n-a}\\
        &= A(a_n+a) + B \leadsto 2Aa + B,
    \end{align*}
    torej je $f$ Cesaro odvedljiva in je $2Ax + B$ njen Cesaro odvod. Opazimo lahko, da je Cesaro odvod enak odvodu.
    \item Naj bo zdaj $f: \mathbb{R} \rightarrow \mathbb{R}$ funkcija s predpisom $f(x) = x^3$. Velja $(-1)^n \leadsto 0$, ampak $$\frac{(-1)^3-0^3}{(-1)-0} = 1 \leadsto 1 \neq 0 = f(0).$$ Torej $f$ ni Cesaro odvedljiva v točki 0.
\end{enumerate}

Vemo že, da je $f: \mathbb{R} \rightarrow \mathbb{R}$ zvezna v točki $a \in \mathbb{R}$ natanko tedaj, ko obstaja limita $f$ v točki $a$ in je ta limita enaka $f(a)$. Podobno velja tudi za Cesaro zveznost, kar nam pove naslednja lema.
\begin{lema}
    Naj bo $f: \mathbb{R} \rightarrow \mathbb{R}$ funkcija. Naj za vsako zaporedje $(a_n)$ s členi različnimi od $0$, ki Cesaro konvergira proti $0$, velja $f(a_n) \leadsto f(0)$. Potem je $f$ Cesaro zvezna v točki $0$.
\end{lema}
\begin{dokaz}
    Predpostaviti smemo, da velja $f(0) = 0$. Podobno kot pri dokazu izreka lahko dokažemo, da velja $f(a+b) = f(a) + f(b)$ za neničelna $a, b \in \mathbb{R}$. Če je katero od obeh števil enako $0$, pa to velja trivialno. Torej je $f$ aditivna. 
    
    Naj bo $(a_n)$ poljubno zaporedje, za katero velja $a_n \leadsto 0$. Pokazati želimo, da velja $f(a_n) \leadsto f(0) = 0$. To bomo storili tako, da bomo skonstruirali zaporedje $(b_n)$ s členi različnimi od $0$, ki se bo malo razlikovalo od zaporedja $(a_n)$, tako da bo veljalo $a_n - b_n \leadsto 0$ in $f(a_n) - f(b_n) \leadsto 0$.

    Ker je množica $\R$ neštevna, množica $\{a_n \mid n \in \N\} \cup \{-a_n \mid n \in \N\}$ pa števna, lahko izberemo $\delta \in \R$, da za vsak $n \in \N$ velja $\delta \neq a_n$ in $\delta \neq -a_n$. Definirajmo $b_n = a_n + (-1)^n \delta$. Za vse $n \in \N$ je $b_n \neq 0$, torej je $(b_n)$ zaporedje z neničelnimi členi. Če je $n$ sod, velja $b_1 + \ldots + b_n = a_1 + \ldots a_n$, če je $n$ lih, pa je $b_1 + \ldots + b_n = a_1 + \ldots a_n - \delta$. Od tod sledi
    $$
    \overline{a}_n - \overline{b}_n = \begin{cases}
        0; & n \ \text{sod}\\
        \frac{\delta}{n}; & n \ \text{lih},
    \end{cases}
    $$
    torej je $\overline{a}_n - \overline{b}_n \rightarrow 0$. Ker zaporedje $(\overline{a}_n)$ konvergira k 0, enako velja za zaporedje $(\overline{b}_n)$, to pa lahko drugače zapišemo kot $b_n \leadsto 0$.

    Zaradi aditivnosti $f$ velja $f(b_n) = f(a_n + (-1)^n \delta) = f(a_n) + f((-1)^n \delta) = f(a_n) + (-1)^n f(\delta)$. Za sod $n$ potem velja $f(b_1) + \ldots + f(b_n) = f(a_1) + \ldots f(a_n)$, za lih $n$ pa $f(b_1) + \ldots + f(b_n) = f(a_1) + \ldots f(a_n) - f(\delta)$. Podobno kot prej sledi
    $$
    \overline{f(a_n)} - \overline{f(b_n)} = \begin{cases}
        0; & n \ \text{sod}\\
        \frac{f(\delta)}{n}; & n \ \text{lih},
    \end{cases}
    $$
    torej je $\overline{f(a_n)} - \overline{f(b_n)} \rightarrow 0$ oziroma $f(a_n) - f(b_n) \leadsto 0$.

    Pokazali smo že, da zaporedje $(b_n)$ Cesaro konvergira k $0$, ker pa ima le neničelne člene, po predpostavki velja $f(b_n) \leadsto f(0) = 0$. Ker velja tudi $f(a_n) - f(b_n) \leadsto 0$, lahko zaključimo, da $(f(a_n))$ cesaro konvergira k $0$, torej je $f$ res Cesaro zvezna v točki $0$.
\end{dokaz}

\begin{izrek}
    Naj bo $f: \mathbb{R} \rightarrow \mathbb{R}$ funkcija. Naslednje trditve so ekvivalentne.
    \begin{enumerate}
        \item Funkcija $f$ je Cesaro odvedljiva v točki $0$.
        \item Funkcija $f$ je Cesaro odvedljiva.
        \item Funkcija $f$ je oblike $f(x) = Ax^2 + Bx + C$ za neka realna števila $A, B, C \in \mathbb{R}$.
    \end{enumerate}
\end{izrek}
\begin{dokaz}
    $(1) \Rightarrow (3): $ Definiramo funkcijo $: \mathbb{R} \rightarrow \mathbb{R}$ s predpisom 
    $$
    g(x) = \begin{cases}
        \frac{f(x)-f(0)}{x}; & x \neq 0\\
        f'(0); & x = 0.
    \end{cases}
    $$
    Naj bo $(a_n)$ zaporedje s členi različnimi od $0$, da velja $a_n \leadsto 0$. Potem velja 
    $$g(a_n) = \frac{f(a_n)-f(0)}{a_n-0} \leadsto f'(0) = g(0),$$
    po definiciji Cesaro odvoda $f$ v točki $0$. Po lemi je $g$ Cesaro zvezna v točki $0$, zato je po izreku oblike $g(x) = Ax + B$ za neka $A, B \in \mathbb{R}$. Potem pa je 
    $f(x) = xg(x) + f(0) = Ax^2 + Bx + C$ za $C = f(0)$.

    $(3) \Rightarrow (2): $ To smo pokazali v zgornjem zgledu.

    $(2) \Rightarrow (1): $ Če je $f$ Cesaro odvedljiva, je po definiciji Cesaro odvedljiva tudi v točki $0$.
\end{dokaz}
Končno smo klasificirali Cesaro zvezne in Cesaro odvedljive funkcije. Opazimo, da obstajajo funkcije, ki so Cesaro odvedljive, ampak niso Cesaro zvezne. Takšna funkcije je na primer $x \mapsto x^2$. To je presenetljivo, saj vemo, da za funkcije iz odvedljivosti sledi zveznost. Pri Cesaro zveznih in Cesaro odvedljivih funkcijah pa temu ni tako.



\section*{Angleško-slovenski slovar strokovnih izrazov}
\geslo{convergence}{konvergenca}
\geslo{continuity}{zveznost}
\geslo{differentiability}{odvedljivost}


\begin{thebibliography}{1}
    \bibitem{1}
    J.~A.~Hocutt in P.~L.~Robinson, \emph{Everywhere Differentiable, Nowhere Continuous Functions}, Amer.~Math.~Monthly \textbf{125} (2018) 923--928.
    \bibitem{2}
    P.~R.~Halmos, \emph{Problems for Mathematicians, Young and Old}, Dolciani Mathematical Expositions \textbf{12}, Mathematical Association of America, Washington, 1991.
\end{thebibliography}


\end{document}