\documentclass[a4paper,12pt]{article}
\usepackage[slovene]{babel}
\usepackage[utf8]{inputenc}
\usepackage[T1]{fontenc}
\usepackage{lmodern}
\usepackage{amsmath,amsfonts}
\usepackage{amsthm}
\usepackage[a4paper, total={17cm, 25cm}]{geometry}

\theoremstyle{definition}
\newtheorem{definicija}{Definicija}
\theoremstyle{plain}
\newtheorem{izrek}{Izrek}

\newenvironment{dokaz}{\begin{proof}[\bfseries\upshape\proofname]}{\end{proof}}

\begin{document}

\section*{Uvod}
V prvem letniku smo spoznali, kako lahko zveznost funkcije v neki točki karakteriziramo z zaporedji. Funkcija $f: \mathbb{R} \rightarrow \mathbb{R}$ je zvezna v točki $a \in \mathbb{R}$ natanko tedaj, ko za vsako zaporedje $(a_n)$, ki konvergira proti $a$, zaporedje $(f(a_n))$ konvergira proti $f(a)$. Podobno lahko z zaporedji karakteriziramo funkcijsko limito: število $L \in \mathbb{R}$ je limita funkcije $f: \mathbb{R} \rightarrow \mathbb{R}$ v točki $a \in \mathbb{R}$ natanko tedaj, ko za vsako zaporedje $(a_n)$ s členi različnimi od $a$, ki konvergira proti $a$, zaporedje $(f(a_n))$ konvergira proti $L$. Po definiciji funkcija $f: \mathbb{R} \rightarrow \mathbb{R}$ odvedljiva v $a$, če obstaja limita $$\lim_{x \rightarrow a} \frac{f(x)-f(a)}{x-a},$$ v tem primeru tej limiti pravimo odvod funkcije $f$ v točki $a$. Ker funkcijsko limito znamo opisati z zaporedji, lahko rečemo, da je $f$ odvedljiva v $a$, natanko tedaj, ko obstaja število $L \in \mathbb{R}$, da za vsako zaporedje $(a_n)$ s členi različnimi od $a$, ki konvergira proti $a$, zaporedje $(\frac{f(x)-f(a_n)}{x-a_n})$ konvergira proti $L$. Sedaj smo uspeli pojma zveznosti in odvedljivosti funkcije opisati samo s pojmom konvergence zaporedja. Če bi znali tudi kako drugače definirati, kdaj dano zaporedje konvergira, bi dobili drugačni definiciji zveznosti in odvedljivosti. Točno s tem se bomo ukvarjali v tej predstavitvi. Najprej se bomo naučili, kaj je to Cesaro konvergenca zaporedja, s pomočjo tega bomo dobili novi definiciji zveznosti in odvedljivosti, nato pa bomo ugotovili katere funkcije so zvezne oziroma odvedljive po tej novi definiciji.



\section{Cesaro konvergenca}
Naj bo $(a_n)$ realno zaporedje. Temu zaporedju lahko priredimo novo zaporedje, katerega n-ti člen je enak $\overline{a}_n = \frac{a_1+a_2+\ldots+a_n}{n}$. Temu novemu zaporedju bomo rekli zaporedje aritmetičnih sredin zaporedja $(a_n)$ in ga označili z $(\overline{a}_n)$.
\begin{definicija}
    Realno zaporedje $(a_n)$ Cesaro konvergira, če konvergira njeno zaporedje aritmetičnih sredin $(\overline{a}_n)$.
\end{definicija}
Oznaka $a_n \rightarrow a$ naj pomeni, da zaporedje $(a_n)$ konvergira proti $a$, oznaka $a_n \leadsto a$ pa, da zaporedje Cesaro konvergira proti $a$. 

V prvem letniku smo se pri analizi naučili, da iz $a_n \rightarrow a$ sledi $a_n \leadsto a$. Obratno seveda ne velja, saj zaporedje $(-1)^n$ Cesaro konvergira proti $0$, ne konvergira pa v običajnem smislu. Ker je pojem Cezaro konvergence zelo pomemben za nadaljevanje, si oglejmo še nekaj primerov.

\subsection{Primeri}
\begin{enumerate}
    \item Poiščimo primer omejenega zaporedja, ki ni Cesaro konvergentno. Prvi člen naj bo enak $2$. Naslednjih nekaj členov bo enakih $-2$. Takih členov mora biti dovolj, da bo aritmetična sredina padla pod $-1$. Nato spet dodajmo dovolj členov enakih $2$, da bo aritmetična sredina narasla nad $1$. S ponavljanjem take kostrukcije dobimo omejeno zaporedje, ki ni Cesaro konvergentno, saj zaporedje aritmetičnih sredin nekako oscilira med $-1$ in $1$.
    \item Naj bo $m \in \mathbb{N}$ in $a_1, a_2, \ldots, a_m \in \mathbb{R}$. Zanima nas, kdaj zaporedje $a_1, a_2, \ldots, a_m, a_1, a_2, \ldots$ Cesaro konvergira proti $0$. Naj bo $A := a_1 + a_2 + \ldots + a_m$. Ker za vse $k \in \mathbb{N}$ velja $\overline{a}_{km} = A$, je enakost $A = 0$ potreben pogoj za $a_n \leadsto 0$. Naj bo torej $A = 0$. Zaporedje delnih vsot zaporedja $(a_n)$ je potem periodično, zato je omejeno. Sledi, da zaporedje aritmetičnih sredin konvergira proti $0$. Torej $(a_n)$ konvergira proti $0$ natanko tedaj, ko velja $A = 0$.
\end{enumerate}



\section{Cesaro zveznost}
Sedaj se lahko končno lotimo Cesaro zveznosti.
\begin{definicija}
    Funkcija $f: \mathbb{R} \rightarrow \mathbb{R}$ je Cesaro zvezna v točki $a \in \mathbb{R}$, če za vsako zaporedje $(a_n)$, ki Cesaro konvergira proti $a$, zaporedje $(f(a_n))$ Cesaro konvergira proti $f(a)$. Pravimo, da je $f$ zvezna, če je zvezna v vsaki točki $a \in \mathbb{R}$.
\end{definicija}
\begin{izrek}
    Naj bo $f: \mathbb{R} \rightarrow \mathbb{R}$ funkcija. Naslednje trditve so ekvivalentne.
    \begin{enumerate}
        \item Funkcija $f$ je Cesaro zvezna v točki $0$.
        \item Funkcija $f$ je Cesaro zvezna.
        \item Funkcija $f$ je oblike $f(x) = Ax + B$ za neki realni števili $A, B \in \mathbb{R}$.
    \end{enumerate}
\end{izrek}
\begin{dokaz}
    $(1) \rightarrow (3):$ Naj bo funkcija $g: \mathbb{R} \rightarrow \mathbb{R}$ definirana s predpisom $g(x) = f(x) - f(0)$. Potem je $g$ Cesaro zvezna v točki $0$ in velja $g(0) = 0$. 
    Naj bo $a \in \mathbb{R}$ poljubno realno število. Ker zaporedje $a, -a, a, -a, a, \ldots$ Cesaro konvergira proti $0$ in je $g$ Cesaro zvezna v $0$, zaporedje $g(a), g(-a), g(a), g(-a), \ldots$ Cesaro konvergira proti $g(0) = 0$. Potem mora veljati $g(a) + g(-a) = 0$ oziroma $g(-a) = -g(a)$. 
    Naj bosta zdaj $b, c \in \mathbb{R}$ poljubni realni števili. Spet zaporedje $b, c, -(b+c), b, c, -(b+c), \ldots$ Cesaro konvergira k $0$, zato zaporedje $g(b), g(c), g(-(b+c)), g(b), \ldots$ Cesaro konvergira proti $g(0) = 0$. Sledi () 
\end{dokaz}


\section{Cesaro odvedljivost}

\end{document}